\documentclass[11pt]{article}
\label{hw1}
\usepackage[utf8]{inputenc}
\usepackage{color, colortbl} % For table coloring
\usepackage{titling} % For positioning of title preamble
\usepackage[margin=1in]{geometry} % For margin width setting
\usepackage{comment} % For block commenting
\usepackage{float} % For table positioning
\usepackage{amsmath} % For math formatting
\usepackage{parskip} % For automatic paragraph spacing/formatting

% Move title area to the top of the page
\setlength{\droptitle}{-4em}
\addtolength{\droptitle}{-4pt} 
\setlength{\tabcolsep}{12pt}
\renewcommand{\arraystretch}{1.25}
% Disable paragraph indenting
\setlength{\parindent}{0pt}

% Custom table column background color
\definecolor{LightCyan}{rgb}{0.88,1,1}
% Change default font to sans font
\renewcommand{\familydefault}{\sfdefault}


\title{CS220 Discrete Math - Homework \#1}
\author{Brendan Nguyen - \texttt{brendan.nguyen001@umb.edu}}
\date{February 3, 2022}

\begin{document}

\maketitle

\section*{Question 1}
The expanded form of the given compound composition is:

\begin{equation*}
    \begin{split}
        \bigwedge_{i=1}^{n-1} \bigwedge_{j=i+1}^n (\neg p_i \vee \neg p_j) &= (\neg p_1 \vee \neg p_2) \wedge (\neg p_1 \vee \neg p_3) \wedge (\neg p_1 \vee \neg p_4)\\
        &\quad\wedge ... \wedge (\neg p_1 \vee \neg p_n) \wedge (\neg p_2 \vee \neg p_3) \wedge ... \wedge (\neg p_2 \vee \neg p_n)\\
        &\quad\wedge ... \wedge (\neg p_{n-1} \vee \neg p_n)
    \end{split}
\end{equation*}
	
Using what we know:

\begin{enumerate}
  \item DeMorgan's Law: $(\neg p_i \vee \neg p_j) = \neg (p_i \wedge p_j)$
  \item The above should be $\texttt{true}$ for all $i$ and $j$
\end{enumerate}

Statement 2 implies that $(p_i \wedge p_j)$ should be $\texttt{false}$ for all $i$ and $j$, which can only be fulfilled for at most 1 $p$ that is $\texttt{true}$. The case of at most 1 $p$ that is $\texttt{true}$ results in $(\neg p_i \vee \neg p_j)$ is always $\texttt{true}$.

\section*{Question 2}

The truth table is as follows:
\begin{table}[H]
\centering
    \begin{tabular}{|c c c|c|c|c|c|c|}
    \hline
    $p$ & $q$ & $r$ & $p \wedge q$ & $\neg r$ & $(p \wedge q) \vee \neg r$ \\ \hline
    T & T & T & T & F & T\\ \hline
    T & T & F & T & T & T\\ \hline
    T & F & T & F & F & F\\ \hline
    T & F & F & F & T & T\\ \hline
    F & T & T & F & F & F\\ \hline
    F & T & F & F & T & T\\ \hline
    F & F & T & F & F & F\\ \hline
    F & F & F & F & T & T\\ \hline
    \end{tabular}
    \caption{Expanded truth table for $(p \wedge q) \vee \neg r$}
    \label{table:1}
\end{table}

\section*{Question 3}

The statement, ``This statement is false," is not a proposition because it cannot have a truth value. If the statement was $\texttt{true}$, then it would assert that it's $\texttt{false}$, which is a contradiction. Similarly, if the statement was $\texttt{false}$, it would assert that it's $\texttt{true}$, another contradiction.

\section*{Question 4}

Testing the truth values of $p$ and $q$ for the compound proposition, you get:

\begin{table}[H]
\centering
    \begin{tabular}{|c c|c|c|c|c|c|}
    \hline
    $p$ & $q$ & $\neg p$ & $p \to q$ & $\neg p \wedge (p \to q)$ & $\neg q$ & $(\neg p \wedge (p \to q)) \to \neg q$\\ \hline
    F & F & T & T & T & T & T\\ \hline
    F & T & T & T & T & F & F\\ \hline
    T & F & F & F & F & T & T\\ \hline
    T & T & F & T & F & F & T\\ \hline
    \end{tabular}
    \caption{Expanded truth table for $(\neg p \wedge (p \to q)) \to \neg q$}
    \label{table:2}
\end{table}

By definition of a tautology, the compound proposition $(\neg p \wedge (p \to q)) \to \neg q$ is not a tautology because not all truth values are $\texttt{true}$ for all $p$ and $q$.

\section*{Question 5}

The simplest way to evaluate the logical equivalence of the compound propositions $(p \to q) \vee (p \to r)$ and $p \to (q \vee r)$ is to compare their truth tables.

\begin{comment}
Combine the 2 following tables and highlight the columns with the compound propositions
\end{comment}

\newcolumntype{g}{>{\columncolor{LightCyan}}c}
\begin{table}[H]
\centering
    \begin{tabular}{|c c c|c|c|g|c|g|}
    \hline
    $p$ & $q$ & $r$ & $p \to q$ & $p \to r$ & $(p \to q) \vee (p \to r)$  & $q \vee r$ & $p \to (q \vee r)$\\ \hline
    T & T & T & T & T & T & T & T\\ \hline
    T & T & F & T & F & T & T & T\\ \hline
    T & F & T & F & T & T & T & T\\ \hline
    T & F & F & F & F & F & F & F\\ \hline
    F & T & T & T & T & T & T & T\\ \hline
    F & T & F & T & T & T & T & T\\ \hline
    F & F & T & T & T & T & T & T\\ \hline
    F & F & F & T & T & T & T & T\\ \hline
    \end{tabular}
    \caption{Expanded truth tables for $(p \to q) \vee (p \to r)$ and $p \to (q \vee (p \to r)$}
    \label{table:3}
\end{table}

% Second compound proposition truth table
\begin{comment}
\begin{table}[H]
\centering
    \begin{tabular}{|c|c|c|c|c|}
    \hline
    $p$ & $q$ & $r$ & $q \vee r$ & $p \to (q \vee r)$\\ \hline
    T & T & T & T & T\\ \hline
    T & T & F & T & T\\ \hline
    T & F & T & T & T\\ \hline
    T & F & F & F & F\\ \hline
    F & T & T & T & T\\ \hline
    F & T & F & T & T\\ \hline
    F & F & T & T & T\\ \hline
    F & F & F & T & T\\ \hline
    \end{tabular}
    \caption{Expanded truth table for $p \to (q \vee (p \to r)$}
    \label{table:4}
\end{table}
\end{comment}

Comparing the 2 truth tables shown in Table \ref{table:3}, you can see that for all $p$, $q$, and $r$, the truth values of the compound propositions are equivalent.

\end{document}  