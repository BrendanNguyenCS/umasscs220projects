\documentclass[letterpaper, 12pt]{article}
\usepackage{nopageno} % For removing page numbers
\usepackage[utf8]{inputenc}
\usepackage{titling} % For positioning of title preamble
\usepackage[margin=0.75in]{geometry} % For margin width setting
\usepackage{comment} % For block commenting
\usepackage{enumitem} % For list styling
\usepackage{float} % For table positioning
% For math equation formatting
\usepackage{amsmath}
\usepackage{relsize}
\usepackage{amssymb}
\usepackage(amsfonts)
% For automatic paragraph spacing/formatting
\usepackage{parskip}
% For side by side figures
\usepackage{multicol}
\usepackage{makecell}
% For colors
\usepackage[dvipsnames]{xcolor}

% Move title area to the top of the page
\setlength{\droptitle}{-4em}
\addtolength{\droptitle}{-4pt} 
% \setlength{\tabcolsep}{12pt}
\renewcommand{\arraystretch}{1.25}
% Disable paragraph indenting
\setlength{\parindent}{0pt}
% Times New Roman font
\usepackage{times}
\renewcommand{\ttdefault}{lmtt}
% Change default vertical spacing for align environments
\setlength{\jot}{7pt}
% Redefine list numbering system
\renewcommand{\labelenumi}{(\alph{enumi})}

\title{CS220 Discrete Math - Homework \#8}
\author{Brendan Nguyen - \texttt{brendan.nguyen001@umb.edu}}
\date{April 7, 2022}

\begin{document}

\maketitle

\section*{Question 1}
First, we consider the properties of the compound proposition of $n$ variables. Each variable in the proposition has 2 options, $\texttt{true}$ or $\texttt{false}$. Therefore, we can use the product rule to say that the amount of options for row will equal two times itself for the number of variables in the proposition $n$, which is $2^n$.

Then, we need to consider that each row will also have 2 options, $\texttt{true}$ or $\texttt{false}$. Applying the product rule again, we see that the amount of options for truth tables will equal two times itself $2^n$ times, which is $2^{2^n}$.

\section*{Question 2}
There are 5 different congruence options for $a$ and 5 different congruence options for $b$. This means that there are 25 classes for the congruence of the pair $(a, b)$. Then, we can use pigeonhole principle to state that:
\[\Big\lceil\frac{n}{25}\Big\rceil = 2\]

Therefore, we can say that \boxed{\mathbf{26}} will guarantee that we have a repeating classes for the congruence of the pair $(a, b)$.

\section*{Question 3}
We can use the pigeonhole principle to prove or disprove the statement: ``Let $n_1,n_2,\ldots,n_t$ be positive integers. If $n_1 + n_2 + \cdots + n_t - t + 1$ objects are placed into $t$ boxes, then for some $i$, $i = 1,2,\ldots,t$, the $i$-th box contains at least $n_i$ objects."

First we can test it by saying that if this hypothesis is not true, then:
\begin{gather*}
    \text{box 1 contains } \leq n_1 - 1\ \text{objects}\\
    \text{box 2 contains } \leq n_2 - 1\ \text{objects}\\
    \vdots\hspace{75pt}\vdots\\
    \text{box $t$ contains } \leq n_t - 1\ \text{objects}
\end{gather*}

Using the equations above, we can say that all the boxes would contain $\leq (n_1 + n_2 + \cdots + n_t) - t$ objects, which would contradict the original hypothesis.

\section*{Question 4}
Given the letters ABCDEFGH,

\begin{enumerate}
    \item The set of permutations of the letters ABCDEFGH that contain the string ED is the same as the set of permutations of the 7-element set \{A, B, C, ED, F, G, H\}. Therefore, there are 7!, or 5040, permutations.
    \item Similarly, the set of permutations that contain CDE is the set of permutations of the 6-element set \{A, B, CDE, F, G, H\}. Therefore, there are 6!, or 720, permutations.
    \item The set of permutations that contains the strings BA and FGH is the set of permutations of the 5-element set \{BA, C, D, E, FGH\}. Therefore, there are 5!, or 120, permutations.
    \item The set of permutations that contains the strings AB, DE, and GH is the set of permutations of the 5-element set \{AB, C, DE, F, GH\}. Therefore, there are also 5!, or 120, permutations of this set.
    \item The only way this is possible is if you have a set that contains CABED. This set of permutations would have elements with \{CABED, F, G, H\}. Therefore, there are 4!, or 24, permutations.
    \item It is impossible for a set of permutations to contain both BCA and ABF.
\end{enumerate}

\section*{Question 5}
The combinations that would satisfy the condition that there must be more women than men in the six person committee is as follows: 4 women and 2 men, 5 women and 1 man, or 6 women.
\begin{align*}
\binom{15}{4} \cdot \binom{10}{2} &= \frac{15!}{4! \cdot 11!} \cdot \frac{10!}{2! \cdot 8!} = 1365 \cdot 45 = 61425\\
\binom{15}{5} \cdot \binom{10}{1} &= \frac{15!}{5! \cdot 10!} \cdot \frac{10!}{1! \cdot 9!} = 3003 \cdot 10 = 30030\\
\binom{15}{6} &= \frac{15!}{6! \cdot 9!} = 5005\\
\binom{15}{4} \cdot \binom{10}{2} + \binom{15}{5} \cdot \binom{10}{1} + \binom{15}{6} &= 61425 + 30030 + 5005 = \boxed{\mathbf{96460}}
\end{align*}

\end{document}