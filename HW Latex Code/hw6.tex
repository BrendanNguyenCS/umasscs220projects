\documentclass[11pt]{article}
\usepackage{nopageno} % For removing page numbers
\usepackage[utf8]{inputenc}
\usepackage{titling} % For positioning of title preamble
\usepackage[margin=1in]{geometry} % For margin width setting
\usepackage{comment} % For block commenting
\usepackage{enumitem} % For list styling
\usepackage{float} % For table positioning
% For math equation formatting
\usepackage{amsmath, amssymb, relsize}
\newcommand{\PMod}[1]{\ (\mathrm{mod}\ #1)}
\newcommand{\Mod}[1]{\ \mathrm{mod}\ #1}
% For automatic paragraph spacing/formatting
\usepackage{parskip}
% For side by side figures
\usepackage{multicol}
\usepackage{makecell}
% For colors
\usepackage[dvipsnames]{xcolor}

% Move title area to the top of the page
\setlength{\droptitle}{-4em}
\addtolength{\droptitle}{-4pt}
% \setlength{\tabcolsep}{12pt}
\renewcommand{\arraystretch}{1.25}
% Disable paragraph indenting
\setlength{\parindent}{0pt}
% Change default font to sans font
\renewcommand{\familydefault}{\sfdefault}
% Change default vertical spacing for align environments
\setlength{\jot}{7pt}

\title{CS220 Discrete Math - Homework \#6}
\author{Brendan Nguyen - \texttt{brendan.nguyen001@umb.edu}}
\date{March 10, 2022}

\begin{document}

\maketitle

\section*{Question 1}
In order to prove or disprove that $p_{1}p_{2} \cdots p_n + 1$ is prime for every $n$ where $p_{1}p_{2} \cdots p_n$ are the $n$ smallest prime numbers, we can test random numbers of the first few prime numbers.
\begin{flalign*}
    n = 1 &\to 2 + 1 = 3 \text{ \textcolor{Green}{prime}}&\\
    2 &\to 2 \cdot 3 + 1 = 7 \text{ \textcolor{Green}{prime}}&\\
    3 &\to 2 \cdot 3 \cdot 5 + 1 = 31 \text{ \textcolor{Green}{prime}}&\\
    4 &\to 2 \cdot 3 \cdot 5 \cdot 7 + 1 = 211 \text{ \textcolor{Green}{prime}}&\\
    5 &\to 2 \cdot 3 \cdot 5 \cdot 7 \cdot 11 + 1 = 2311 \text{ \textcolor{Green}{prime}}&\\
    6 &\to 2 \cdot 3 \cdot 5 \cdot 7 \cdot 11 \cdot 13 + 1 = 30031 = 59 \cdot 509 \text{ \textcolor{red}{NOT prime}}
\end{flalign*}

Because the statement fails for the 6 smallest prime numbers, we can say that $p_{1}p_{2} \cdots p_n + 1$ is not prime for every $n$.

\section*{Question 2}
First, we need to find $\gcd(34, 89)$ using the Euclidean algorithm.
\begin{align*}
    89 &= 2 \cdot 34 + 21\\
    34 &= 1 \cdot 21 + 13\\
    21 &= 1 \cdot 13 + 8\\
    13 &= 1 \cdot 8 + 5\\
    8 &= 1 \cdot 5 + 3\\
    5 &= 1 \cdot 3 + 2\\
    3 &= 1 \cdot 2 + \boxed{\mathbf{1}}
\end{align*}

Then we use B\'{e}zout's theorem to find the linear combination. We will work backwards from the previous operations.

\begin{align*}
    1 &= 3 - (5 - 3)\\
    &= 2 \cdot 3 - 5\\
    &= 2 \cdot (8 - 5) - 5\\
    &= 2 \cdot 8 - 3 \cdot 5\\
    &= 2 \cdot 8 - 3 \cdot (13 - 8)\\
    &= 5 \cdot 8 - 3 \cdot 13\\
    &= 5 \cdot (21 - 13) - 3 \cdot 13\\
    &= 5 \cdot 21 - 8 \cdot 13\\
    &= 5 \cdot 21 - 8 (34 - 21)\\
    &= 13 \cdot 21 - 8 \cdot 34\\
    &= 13 \cdot (89 - 2 \cdot 34) - 8 \cdot 34\\
    &= 13 \cdot 89 - 26 \cdot 34 - 8 \cdot 34\\
    &= 13 \cdot 89 - \boxed{\mathbf{34}} \cdot 34
\end{align*}

The inverse of 34 modulo 89 is \textbf{-34} or \textbf{55}.

\section*{Question 3}
Using $a$ that we found previously, we can solve the given linear congruence.

\begin{align*}
    34x &\equiv 77 \PMod{89}\\
    1870x &= 55 \cdot 77 \PMod{89}\\
    x &\equiv 4235 \equiv 47 \cdot 89 + 52 \equiv \boxed{\mathbf{52}}\PMod{89}
\end{align*}

\section*{Question 4}
The pairs of positive integers that are less than 11 (that don't include 1 or 10) such that each pair are inverses of each other modulo 11 are shown below:

\begin{align*}
    2 \cdot 6 = 12 = 1\cdot 11 + 1 \equiv 1 \PMod{11} \\
    3 \cdot 4 = 12 = 1 \cdot 11 + 1 \equiv 1 \PMod{11}\\
    5 \cdot 9 = 45 = 4 \cdot 11 + 1 \equiv 1 \PMod{11}\\
    7 \cdot 8 = 56 = 5 \cdot 11 + 1 \equiv 1 \PMod{11}
\end{align*}

\section*{Question 5}
Fermat's Little Theorem states that, if a number $p$ is prime and another number $a$ is not divisible by $p$, then

\[a^{(p-1)} = 1 \PMod{p}\]

Therefore, we can solve $23^{1002} \PMod{41}$ by:

\begin{align*}
    23^{1002} \PMod{41} &= (23^{40})^{23} \cdot 23^2 \PMod{41}\\
    &= 1^{23} \cdot 23^2 \PMod{41}\\
    &= 23^2 \PMod{41}\\
    &= 529 \PMod{41}\\
    529 &= 12 \cdot 41 + 37\\
    529 \PMod{41} &= \boxed{\mathbf{37}}
\end{align*}

\end{document}