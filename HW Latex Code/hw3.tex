\documentclass[11pt]{article}
\usepackage{nopageno} % For removing page numbers
\usepackage[utf8]{inputenc}
\usepackage{titling} % For positioning of title preamble
\usepackage[margin=1in]{geometry} % For margin width setting
\usepackage{comment} % For block commenting
\usepackage{enumitem} % For list styling
\usepackage{float} % For table positioning
% For math symbol formatting and other math tools
\usepackage{mathtools}
% For math equation formatting
\usepackage{amsmath, amssymb, relsize}
\usepackage{parskip} % For automatic paragraph spacing/formatting

% Move title area to the top of the page
\setlength{\droptitle}{-4em}
\addtolength{\droptitle}{-4pt} 
\setlength{\tabcolsep}{12pt}
\renewcommand{\arraystretch}{1.25}
% Increase width between table columns
\setlength{\tabcolsep}{1cm}
\setlength{\parskip}{1em}
% Change default font to sans font
\renewcommand{\familydefault}{\sfdefault}
% Change default vertical spacing for align environments
\setlength{\jot}{7pt}

\title{CS220 Discrete Math - Homework \#3}
\author{Brendan Nguyen - \texttt{brendan.nguyen001@umb.edu}}
\date{February 17, 2022}

\begin{document}

\maketitle

\section*{Question 1}
$A$, $B$, and $C$ are sets.
\begin{flalign*}
    (A - C) - (B - C) & = (A \cap C^c) \cap (B \cap C^c)^c&\\
    & = (A \cap C^c) \cap (B^c \cup C)&\\
    & = ((A \cap C^c) \cap B^c) \cup ((A \cap C^c) \cap C)&\\
    & = ((A \cap B^c) \cap C^c ) \cup (A \cap (C^c \cap C))&\\
    & = ((A \cap B^c) \cap C^c) \cup (A \cap \varnothing)&\\
    & = ((A - B) - C) \cup \varnothing&\\
    & = (A - B) - C
\end{flalign*}

\section*{Question 2}
\newcommand*{\defeq}{\stackrel{\mathsmaller{\mathsf{def}}}{=}}

By definition, $f(x)$ is strictly increasing if:
\[\forall x \forall y (x < y \to f(x) < f(y))\]
Dividing the inequality $f(x) < f(y)$ by the inequality $f(x)f(y) > 0$ results in:
\[\frac{1}{f(y)} < \frac{1}{f(x)}\]
The above inequality is equal to $g(y) < g(x)$, therefore:
\[\forall x \forall y (x < y \to g(x) > g(y))\]
Conversely, we can prove the inverse by testing $g(x) = \frac{1}{f(x)}$ which is strictly decreasing:
\[\forall x \forall y (x < y \to g(x) > g(y))\]
Using $g(x) > g(y) \defeq \frac{1}{f(x)} < \frac{1}{f(y)}$ that we proved previously, we get:
\[\forall x \forall y (x < y \to f(x) < f(y))\]
Meaning that $f(x)$ is strictly increasing.

\section*{Question 3}
\renewcommand{\labelenumi}{(\alph{enumi})}

\begin{enumerate}
    \item $A_n = 1.09 \cdot A_{n-1}$ denotes the recurrence relation for the amount in the account at the end of $n$ years.
    \item $A_n = 1000 \cdot 1.09^n$ denotes the explicit formula for the amount in the account at the end of $n$ years.
    \item $A_{100} = 1000 \cdot 1.09^{100} = \$ 5,529,040.79$ is the amount of money in the account after 100 years.
\end{enumerate}

\section*{Question 4}
\begin{flalign*}
    \sum_{i=1}^{n} \frac{1}{i(i+1)} & = \sum_{k=1}^{n} \left(\frac{1}{k}-\frac{1}{k+1}\right)&\\
    & = \frac{1}{1} - \frac{1}{2} + \frac{1}{2} - \frac{1}{3} + \frac{1}{3} - \frac{1}{4} + \cdots + \frac{1}{n} - \frac{1}{n+1}&\\
    & = 1 - \frac{1}{n+1}
\end{flalign*}

\section*{Question 5}
To show that the set of functions $\{ 0, 1, 2, 3, 4, 5, 6, 7, 8, 9 \}$ is uncountable, we can use the fact that the set of all subsets of $\mathbb{N}$, $F(\mathbb{N})$, is uncountable. We see that the set of functions from $\mathbb{N}$ to $\{ 0, 1, 2, 3, 4, 5, 6, 7, 8, 9 \}$ contains the set $\{0, 1\}^{\mathbb{N}}$ of functions from $\mathbb{N}$ to $\{0,1\}$ using injection. Therefore, you can say that there is a bijection between $F(\mathbb{N})$ and $\{0, 1\}^{\mathbb{N}}$. In conclusion, since the set $\{0,1\}^{\mathbb{N}}$ is uncountable and the set is a subset in the set $\{ 0, 1, 2, 3, 4, 5, 6, 7, 8, 9 \}$, then we can say that the set $\{ 0, 1, 2, 3, 4, 5, 6, 7, 8, 9 \}$ is also uncountable.

\end{document}