\documentclass[11pt]{article}
\usepackage[utf8]{inputenc}
\usepackage{titling} % For positioning of title preamble
\usepackage[margin=1in]{geometry} % For margin width setting
\usepackage{comment} % For block commenting
\usepackage{enumitem} % For list styling
\usepackage{float} % For table positioning
% For math equation formatting
\usepackage{amsmath, amssymb, relsize}
% For automatic paragraph spacing/formatting
\usepackage{parskip}
% For side by side figures
\usepackage{multicol}
\usepackage{makecell}
% For colors
\usepackage[dvipsnames]{xcolor}
% For code
\usepackage{listings}
\lstdefinestyle{mystyle}{
    basicstyle=\ttfamily\footnotesize,
    breakatwhitespace=false,         
    breaklines=true,                 
    captionpos=b,                    
    keepspaces=true,                 
    numbersep=5pt,                  
    showspaces=false,                
    showstringspaces=false,
    showtabs=false,                  
    tabsize=2
}
\lstset{style=mystyle}


% Move title area to the top of the page
\setlength{\droptitle}{-4em}
\addtolength{\droptitle}{-4pt} 
% \setlength{\tabcolsep}{12pt}
\renewcommand{\arraystretch}{1.25}
% Disable paragraph indenting
\setlength{\parindent}{0pt}
% Change default font to sans font
\renewcommand{\familydefault}{\sfdefault}

\title{CS220 Discrete Math - Homework \#9}
\author{Brendan Nguyen - \texttt{brendan.nguyen001@umb.edu}}
\date{April 14, 2022}

\begin{document}

\maketitle

\section*{Question 1}


\section*{Question 2}
Using a combinatorial argument, we will assume that there are two groups A and B, each of which has $n$ members in them. If we select 2 members in total, we can select either:

\begin{itemize}
    \item One from each group: $\binom{n}{1} \cdot \binom{n}{1}$
    \item Two from either group: $\binom{n}{2} + \binom{n}{2}$
\end{itemize}

Therefore, we can say that:

\begin{flalign*}
    \binom{2n}{2} &= \binom{n}{1} \cdot \binom{n}{1} + \binom{n}{2} + \binom{n}{2}\\
    &= n \cdot n + 2\binom{n}{2}\\
    &= 2\binom{n}{2} + n^2\\
\end{flalign*}

\section*{Question 3}
If a standard deck of 52 cards is a dealt to four players, each player will get 13, or $\frac{52}{4}$, cards. Then we can use the theorem for the number of permutations with distinguishable objects which allow us to compute the number of permutations of $n$ objects in $k$ different categories given by:

\[\frac{n!}{n_1!n_2! \cdots n_k!}\]

In this example, we know that $n = 52$, $k = 4$, and $n_1, n_2, n_3, n_4 = 13$. Applying these values to the previously mentioned theorem, we get:

\[\frac{52!}{13! \cdot 13! \cdot 13! \cdot 13!} = \frac{52!}{(13!)^4} \approx 5.4 \cdot 10^{28} \text{ permutations}\]

\section*{Question 4}


\section*{Question 5}
\renewcommand{\labelenumi}{(\alph{enumi})}

\begin{enumerate}
    \item In order to find the number of ways to place 5 distinguishable balls into 7 distinguishable boxes, we have to find the number of permutations since order matters, which is given by: 
    
    \[P(n, k) = \frac{n!}{(n-k)!}\]
    
    For our problem, $n$ is the number of distinguishable boxes, 7, and $k$ is the number of distinguishable balls, 5.
    
    \[P(7, 5) = \frac{7!}{(7-5)!} = \frac{7!}{2!} = 7 \cdot 6 \cdot 5 \cdot 4 \cdot 3 = 2520 \text{ permutations}\]
    
    \item In order to find the number of ways to place 5 indistinguishable balls into 7 distinguishable boxes, we have to find the number of combinations since order doesn't matter, which is given by:
    
    \[C(n, k) = \binom{n}{k} = \frac{n!}{k!(n-k)!}\]
    
    As with the previous part, $n$ is the number of distinguishable boxes, 7, but $k$ is the number of indistinguishable balls, 5.
    
    \[\binom{7}{5} = \frac{7!}{5!(7-5)!} = \frac{7!}{5!2!} = 21 \text{ combinations}\]
\end{enumerate}

\end{document}