\documentclass[11pt]{article}
\usepackage[utf8]{inputenc}
\usepackage{titling} % For positioning of title preamble
\usepackage[margin=1in]{geometry} % For margin width setting
\usepackage{comment} % For block commenting
\usepackage{enumitem} % For list styling
\usepackage{float} % For table positioning
% For math equation formatting
\usepackage{amsmath, amssymb, relsize}
% For automatic paragraph spacing/formatting
\usepackage{parskip}
% For side by side figures
\usepackage{multicol}
\usepackage{makecell}
% For colors
\usepackage[dvipsnames]{xcolor}
% For code
\usepackage{listings}
\lstdefinestyle{mystyle}{
    basicstyle=\ttfamily\footnotesize,
    breakatwhitespace=false,         
    breaklines=true,                 
    captionpos=b,                    
    keepspaces=true,                 
    numbersep=5pt,                  
    showspaces=false,                
    showstringspaces=false,
    showtabs=false,                  
    tabsize=2
}
\lstset{style=mystyle}


% Move title area to the top of the page
\setlength{\droptitle}{-4em}
\addtolength{\droptitle}{-4pt} 
% \setlength{\tabcolsep}{12pt}
\renewcommand{\arraystretch}{1.25}
% Disable paragraph indenting
\setlength{\parindent}{0pt}
% Change default font to sans font
\renewcommand{\familydefault}{\sfdefault}
% Change default vertical spacing for align environments
\setlength{\jot}{7pt}

\title{CS220 Discrete Math - Homework \#11}
\author{Brendan Nguyen - \texttt{brendan.nguyen001@umb.edu}}
\date{April 28, 2022}

\begin{document}

\maketitle

\section*{Question 1}


\section*{Question 2}
Given that $E$, $F_1$, $F_2$, and $F_3$ are events from a sample space $S$ and that $F_1$, $F_2$, and $F_3$ are pairwise disjoint with their union $S$, we can use the Generalized Bayes' Theorem to find $p(F_2|E)$. The theorem is given by:

\[p(F_j|E) = \frac{p(E|F_j)p(F_j)}{\sum^{n}_{i=1}p(E|F_i)p(F_i)}\]

Applying this to our problem, we get:

\[p(F_2|E) = \frac{p(E|F_2)p(F_2)}{p(E|F_1)p(F_1) + p(E|F_2)p(F_2) + p(E|F_3)p(F_3)}\]

Finally, we just need to insert all of the values of the equation. We know that: $p(E|F_1) = \frac{2}{7}$, $p(E|F_2) = \frac{3}{8}$, $p(E|F_3) = \frac{1}{2}$, $p(F_1) = \frac{1}{6}$, $p(F_2) = \frac{1}{2}$, and $p(F_3) = \frac{1}{3}$. So, $p(F_2|E)$ is:

\begin{flalign*}
p(F_2|E) &= \frac{p(E|F_2)p(F_2)}{p(E|F_1)p(F_1) + p(E|F_2)p(F_2) + p(E|F_3)p(F_3)}\\
&= \frac{\left(\frac{3}{8}\right)\left(\frac{1}{2}\right)}{\left(\frac{2}{7}\right)\left(\frac{1}{6}\right) + \left(\frac{3}{8}\right)\left(\frac{1}{2}\right) + \left(\frac{1}{2}\right)\left(\frac{1}{3}\right)}\\
&= \frac{\frac{3}{16}}{\frac{1}{21} + \frac{3}{16} + \frac{1}{6}}  = \frac{\frac{3}{16}}{\frac{79}{336} + \frac{56}{336}}\\
&= \frac{\frac{3}{16}}{\frac{45}{112}} = \left(\frac{3}{16}\right)\left(\frac{112}{45}\right)\\
&= \frac{7}{15}
\end{flalign*}

\section*{Question 3}
In order to compute the expected value of a \$1 lottery ticket, you need to compute a number of two elements: the probability of winning and the probability of losing. First, we know that there are $\binom{50}{6}$, or 15,890,700, possible tickets. Therefore:

\[P(\text{winning}) = \frac{1}{15,890,700} \approx 6.29 \cdot 10^{-8}\]

From this, we also know that the probability of losing is $1- P(\text{winning})$, or 99\%. A person with a winning ticket would have a net gain of \$9,999,999 but a person with a losing ticket would have a net loss of \$1. Therefore, the expected value $E(t)$ for an imaginary function $f(t)$ which relates ticket to winnings can be calculated as:

\begin{flalign*}
E(t) &= f(\text{winning})P(\text{winning}) + f(\text{losing})P(\text{losing})\\
&= 9,999,999 \cdot P(\text{winning}) - 1 \cdot P(\text{losing})\\
&= \frac{9,999,999}{\binom{50}{6}} - \left(1 - \frac{1}{\binom{50}{6}}\right)\\
&= \frac{10,000,000}{\binom{50}{6}} - 1\\
&= \frac{10,000,000}{15,890,700} - 1\\
&= 0.629 - 1\\
&= -0.371
\end{flalign*}

Our expected value of -0.371 means that a person loses 37 cents on average every time you pay the \$1 price for a lottery ticket.

\section*{Question 4}
To find the variance of the number of times a 6 appears when a fair die is rolled 10 times, we can use the variance for a binomial distribution which is given by:

\[\sigma^2 = np(1-p)\]

Where $n$ is the number of trials and $p$ is the probability of getting a success in a fair trial. In this situation, we have 10 trials with the probability of rolling a 6, the successful result, being $\frac{1}{6}$. Therefore, the variance is:

\[\sigma^2 = np(1-p) = (10)\left(\frac{1}{6}\right)\left(1- \frac{1}{6}\right) = (10)\left(\frac{1}{6}\right)\left(\frac{5}{6}\right) = \frac{25}{18} \approx 1.39\]

\section*{Question 5}


\end{document}