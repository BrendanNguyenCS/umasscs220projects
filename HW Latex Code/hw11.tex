\documentclass[letterpaper, 12pt]{article}
\usepackage{nopageno} % For removing page numbers
\usepackage[utf8]{inputenc}
\usepackage{titling} % For positioning of title preamble
\usepackage[margin=0.75in]{geometry} % For margin width setting
\usepackage{comment} % For block commenting
\usepackage{enumitem} % For list styling
\usepackage{float} % For table positioning
% For math equation formatting
\usepackage{amsmath, amssymb, relsize}
% For automatic paragraph spacing/formatting
\usepackage{parskip}
% For side by side figures
\usepackage{multicol}
\usepackage{makecell}
% For colors
\usepackage[dvipsnames]{xcolor}


% Move title area to the top of the page
\setlength{\droptitle}{-4em}
\addtolength{\droptitle}{-4pt} 
% \setlength{\tabcolsep}{12pt}
\renewcommand{\arraystretch}{1.25}
% Disable paragraph indenting
\setlength{\parindent}{0pt}
% Times New Roman font
\usepackage{times}
\renewcommand{\ttdefault}{lmtt}
% Change default vertical spacing for align environments
\setlength{\jot}{7pt}

\title{CS220 Discrete Math - Homework \#11}
\author{Brendan Nguyen - \texttt{brendan.nguyen001@umb.edu}}
\date{April 28, 2022}

\begin{document}

\maketitle

\section*{Question 1}
Given that $p(E) = \frac{2}{3}$, $p(F) = \frac{3}{4}$, and $p(F|E) = \frac{5}{8}$, we can solve for $p(E|F)$ by calculating the union of $E$ and $F$.
\begin{align*}
p(F|E) &= \frac{p(E \cap F)}{p(E)}\\
p(E \cap F) &= p(F|E)p(E)\\
&= \left(\frac{5}{8}\right)\left(\frac{2}{3}\right)\\
&= \frac{5}{12} \approx 0.417
\end{align*}

Finally, all we have to do is solve for $p(E|F)$ similarly to how to solved the above.
\begin{align*}
p(E|F) &= \frac{p(E \cap F)}{p(F)}\\
&= \left(\frac{5}{12}\right)\left(\frac{4}{3}\right)\\
&= \frac{5}{9} \approx 0.556
\end{align*}

\section*{Question 2}
Given that $E$, $F_1$, $F_2$, and $F_3$ are events from a sample space $S$ and that $F_1$, $F_2$, and $F_3$ are pairwise disjoint with their union $S$, we can use the Generalized Bayes' Theorem to find $p(F_2|E)$. The theorem is given by:
\[p(F_j|E) = \frac{p(E|F_j)p(F_j)}{\sum^{n}_{i=1}p(E|F_i)p(F_i)}\]

Applying this to our problem, we get:
\[p(F_2|E) = \frac{p(E|F_2)p(F_2)}{p(E|F_1)p(F_1) + p(E|F_2)p(F_2) + p(E|F_3)p(F_3)}\]

Finally, we just need to insert all of the values of the equation. We know that: $p(E|F_1) = \frac{2}{7}$, $p(E|F_2) = \frac{3}{8}$, $p(E|F_3) = \frac{1}{2}$, $p(F_1) = \frac{1}{6}$, $p(F_2) = \frac{1}{2}$, and $p(F_3) = \frac{1}{3}$. So, $p(F_2|E)$ is:

\begin{align*}
p(F_2|E) &= \frac{p(E|F_2)p(F_2)}{p(E|F_1)p(F_1) + p(E|F_2)p(F_2) + p(E|F_3)p(F_3)}\\
&= \frac{\left(\frac{3}{8}\right)\left(\frac{1}{2}\right)}{\left(\frac{2}{7}\right)\left(\frac{1}{6}\right) + \left(\frac{3}{8}\right)\left(\frac{1}{2}\right) + \left(\frac{1}{2}\right)\left(\frac{1}{3}\right)}\\
&= \frac{\frac{3}{16}}{\frac{1}{21} + \frac{3}{16} + \frac{1}{6}}\\
&= \frac{\frac{3}{16}}{\frac{45}{112}} = \left(\frac{3}{16}\right)\left(\frac{112}{45}\right)\\
&= \frac{7}{15} \approx 0.467
\end{align*}

\section*{Question 3}
In order to compute the expected value of a \$1 lottery ticket, you need to compute a number of two elements: the probability of winning and the probability of losing. First, we know that there are $\binom{50}{6}$, or 15,890,700, possible tickets. Therefore:
\[P(\text{winning}) = \frac{1}{15,890,700} \approx 6.29 \cdot 10^{-8}\]

From this, we also know that the probability of losing is $1- P(\text{winning})$, or 99\%. A person with a winning ticket would have a net gain of \$9,999,999 but a person with a losing ticket would have a net loss of \$1. Therefore, the expected value $E(t)$ for an imaginary function $f(t)$ which relates ticket to winnings can be calculated as:
\begin{flalign*}
E(t) &= f(\text{winning})P(\text{winning}) + f(\text{losing})P(\text{losing})\\
&= 9,999,999 \cdot P(\text{winning}) - 1 \cdot P(\text{losing})\\
&= \frac{9,999,999}{\binom{50}{6}} - \left(1 - \frac{1}{\binom{50}{6}}\right)\\
&= \frac{10,000,000}{\binom{50}{6}} - 1\\
&= \frac{10,000,000}{15,890,700} - 1\\
&= 0.629 - 1\\
&\approx -0.371
\end{flalign*}

Our expected value of -0.371 means that a person loses 37 cents on average every time you pay the \$1 price for a lottery ticket.

The wording of the question can also be understood as the expected value of the ticket given that you have already bought the ticket. This changes the expected value to not include the value of a losing ticket (which would be \$0). Therefore, the expected value just be:
\begin{align*}
E(t) &= f(\text{winning})P(\text{winning})\\
&= \frac{10,000,000}{\binom{50}{6}}\\
&\approx 0.629
\end{align*}

This expected value of 0.629 means that a ticket buyer is expected to win 63 cents on average for every ticket.

\section*{Question 4}
To find the variance of the number of times a 6 appears when a fair die is rolled 10 times, we can use the variance for a binomial distribution which is given by:
\[\sigma^2 = np(1-p)\]

Where $n$ is the number of trials and $p$ is the probability of getting a success in a fair trial. In this situation, we have 10 trials with the probability of rolling a 6, the successful result, being $\frac{1}{6}$. Therefore, the variance is:
\[\sigma^2 = npq = np(1-p) = (10)\left(\frac{1}{6}\right)\left(1- \frac{1}{6}\right) = (10)\left(\frac{1}{6}\right)\left(\frac{5}{6}\right) = \frac{25}{18} \approx 1.39\]

\section*{Question 5}
Chebyshev's Inequality tells us that:
\[p(\left|X(s) - E(X)\right| \geq r) \leq \frac{V(X)}{r^2}\]

In our problem, we will be finding the upper bound on the probability that the deviation of the coin flip is greater than $\sqrt{n}$. We can treat this problem as a Bernoulli experiment where the probability of success (getting a tail) is $\frac{2}{5}$, or 0.4. 

By Theorem 4.2, the expected value for mutually independent Bernoulli trials is $np$, which is $E(X) = n \cdot \frac{2}{5} = \frac{2n}{5}$ in this case. Similarly, the variance of mutually independent Bernoulli trials is $np(1-p)$, or $V(X) = n \cdot \frac{3}{5} \cdot \frac{2}{5} = \frac{6n}{25}$. Finally, we can plug in all these values in addition to $r = \sqrt{n}$:
\[p\left(\left|X(s) - \frac{2n}{5}\right| \geq \sqrt{n} \right) \leq \frac{\frac{6n}{25}}{{(\sqrt{n})}^2} = \frac{6}{25} = 0.24\]

\end{document}