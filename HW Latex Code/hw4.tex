\documentclass[letterpaper, 12pt]{article}
\usepackage{nopageno} % For removing page numbers
\usepackage[utf8]{inputenc}
\usepackage{titling} % For positioning of title preamble
\usepackage[margin=0.75in]{geometry} % For margin width setting
\usepackage{comment} % For block commenting
\usepackage{enumitem} % For list styling
\usepackage{float} % For table positioning
% For math symbol formatting and other math tools
\usepackage{mathtools}
% For math equation formatting
\usepackage{amsmath, amssymb, relsize}
% For code blocks
\usepackage{listings}
\lstdefinestyle{mystyle}{
    basicstyle=\ttfamily\small,
    breakatwhitespace=false,         
    breaklines=true,                 
    captionpos=b,                    
    keepspaces=true,                 
    numbersep=5pt,                  
    showspaces=false,                
    showstringspaces=false,
    showtabs=false,                  
    tabsize=2
}
\lstset{style=mystyle}
% For automatic paragraph spacing/formatting
\usepackage{parskip}
% Change default font to sans font
\renewcommand{\familydefault}{\sfdefault}

% Move title area to the top of the page
\setlength{\droptitle}{-4em}
\addtolength{\droptitle}{-4pt} 
\setlength{\tabcolsep}{12pt}
\renewcommand{\arraystretch}{1.25}
% Disable paragraph indenting
\setlength{\parindent}{0pt}
% Increase width between table columns
\setlength{\tabcolsep}{1cm}
% Redefine list numbering system
\renewcommand{\labelenumi}{(\alph{enumi})}

\title{CS220 Discrete Math - Homework \#4}
\author{Brendan Nguyen - \texttt{brendan.nguyen001@umb.edu}}
\date{February 24, 2022}

\begin{document}

\maketitle

\section*{Question 1}
The definition of big-$\mathrm{O}$ notation tells you that have you to find the witnesses $c$ and $k$ such that $f(x) \leq c(g(x))$ when $f(x) = \mathrm{O}(g(x))$.

First, I tested $c=2$ and $k=2$. Testing the definition for $n=3$, I got $f(2.5) \leq c(g(2.5)) \to 25 \leq 54$. Although this pair seemed to work, when testing values approaching $n=2$, I found that $n=2.1$ fails the inequality ($f(2.1)=21.287$ and $2(g(2.1))=20.090$).

Afterwards, I tested $c=2$ and $k=3$. Plugging in $n=4$, I got $f(4) \leq c(g(4)) \to 33 \leq 162$. Testing values approaching $n=3$ shows us that the $\textbf{witnesses}$ $c=2$ and $k=3$ are valid as no decimals fail the inequality.

\section*{Question 2}
According to the definition of big-$\mathrm{O}$ notation, we can say that:

\[1^k + 2^k + \cdots + n^k < n^k + n^k + \cdots + n^k = n \cdot n^k = n^{k+1}\]

Since the sum $(n^k + n^k + \cdots + n^k)$ is greater than the sum $(1^k + 2^k + \cdots + n^k)$ and is clearly $\mathrm{O}(n^{k+1})$ since the sum is exactly $n^{k+1}$. Therefore the smaller sum of $(1^k + 2^k + \cdots + n^k)$ is also $\mathrm{O}(n^{k+1})$.

\section*{Question 3}
There are some statements that we can say about the big-$\mathrm{O}$ estimates involved in the problem:
\begin{enumerate}[label=\arabic*.]
    \item Logarithmic functions grow slower than all positive powers of $n$ (i.e. $\sqrt{n}$, $n$, $n^2$, $n^3$, etc.)
    \item Exponential functions grow faster than polynomial functions
    \item Factorials grow faster than exponential functions
\end{enumerate}

With these statements in mind, we can order the given functions as such:

\[(\log{n})^3, \sqrt{n}\log{n}, n^{99} + n^{98}, n^{100}, (1.5)^n, 10^n, (n!)^2\]

We use Statement 1 to say that $(\log{n})^3$ is the slowest growing function. The next three can be placed in order of power of $n$ due to Statement 2 ($\frac{1}{2}$, 99, and 100, respectively). Statement 2 also puts the two exponential functions afterwards in order of base. Finally, the statement $(n!)^2$ is the fastest growing function due to Statement 3.

\section*{Question 4}
The C code block from page 50 and Exercise 2-9 of the C Programming Language by Kernighan and Ritchie, the second edition.

\begin{lstlisting}[language=C]
int bitCount(unsigned x) {
    int count;
    
    for (count = 0; x != 0; x &= (x - 1))
        count++;
    return count;
}
\end{lstlisting}

\begin{enumerate}
    \item The best way to show that the above function returns the number of 1 bits in the unsigned integer $\texttt{x}$ is to use an example of the computation of unsigned int $\texttt{x = 7}$.
    
    
    The first iteration of the loop would compare $\texttt{x = 7}$ and $\texttt{x = 6}$.
    \[\begin{array}[t]{r}
        \texttt{0111}\\
    \texttt{\&} \ \texttt{0110}\\ \hline
        \texttt{0110}
    \end{array}\]
    
    The second iteration has a $\texttt{count = 1}$ with $\texttt{x = 6}$ and it would compare $\texttt{x = 6}$ and $\texttt{x = 5}$.
    \[\begin{array}[t]{r}
        \texttt{0110}\\
    \texttt{\&} \ \texttt{0101}\\ \hline
        \texttt{0100}
    \end{array}\]
    
    The third iteration has a $\texttt{count = 2}$ with $\texttt{x = 4}$ and it would compare $\texttt{x = 4}$ and $\texttt{x = 3}$.
    \[\begin{array}[t]{r}
        \texttt{0100}\\
    \texttt{\&} \ \texttt{0011}\\ \hline
        \texttt{0000}
    \end{array}\]
    
    Now the $\texttt{count = 3}$ and $\texttt{x = 0}$ so the loop would end. The $\texttt{count}$ now equals the number of 1 bits in the unsigned integer 7 ($\texttt{0111}$).
    
    \item The number of iterations equals the number of $\texttt{count}$ which equals the number of 1 bits in the unsigned integer $\texttt{x}$.
\end{enumerate}

\section*{Question 5}
We can find which method is more efficient by analyzing the matrix multiplication operations for each method. In the case of $\texttt{(AB)C}$, you calculate the number of operations in $\texttt{AB}$ and add it to the operations of multiplying $\texttt{C}$ and the resulting matrix of $\texttt{AB}$. Similarly for $\texttt{A(BC)}$, you calculate the number of operations in $\texttt{BC}$ and add it to the operations of multiplying $\texttt{A}$ and the resulting matrix of $\texttt{BC}$.

For the matrices $A$, $B$, and $C$ with dimensions $3 \times 9$, $9 \times 4$, and $4 \times 2$, respectively:

\[\texttt{(AB)C} = (3 \cdot 9 \cdot 4) + (3 \cdot 4 \cdot 2) = 132 \ \text{integer multiplications}\]
\[\texttt{A(BC)} = (9 \cdot 4 \cdot 2) + (3 \cdot 9 \cdot 2) = 126 \ \text{integer multiplications}\]

The above calculations show that $\texttt{A(BC)}$ is more efficient than $\texttt{(AB)C}$ while maintaining the resulting $3 \times 2$ matrix.

\end{document}