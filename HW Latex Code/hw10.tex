\documentclass[11pt]{article}
\usepackage{nopageno} % For removing page numbers
\usepackage[utf8]{inputenc}
\usepackage{titling} % For positioning of title preamble
\usepackage[margin=1in]{geometry} % For margin width setting
\usepackage{comment} % For block commenting
\usepackage{enumitem} % For list styling
\usepackage{float} % For table positioning
% For math equation formatting
\usepackage{amsmath, amssymb, relsize}
% For automatic paragraph spacing/formatting
\usepackage{parskip}
% For side by side figures
\usepackage{multicol}
\usepackage{makecell}
% For colors
\usepackage[dvipsnames]{xcolor}


% Move title area to the top of the page
\setlength{\droptitle}{-4em}
\addtolength{\droptitle}{-4pt} 
% \setlength{\tabcolsep}{12pt}
\renewcommand{\arraystretch}{1.25}
% Disable paragraph indenting
\setlength{\parindent}{0pt}
% Change default font to sans font
\renewcommand{\familydefault}{\sfdefault}

\title{CS220 Discrete Math - Homework \#10}
\author{Brendan Nguyen - \texttt{brendan.nguyen001@umb.edu}}
\date{April 21, 2022}

\begin{document}

\maketitle

\section*{Question 1}
To get the probability that a five-card poker hand contains a royal flush (a 10, jack, queen, king, and ace of the same suit), we first consider the amount of total possible five-card poker hands which is:
\[\binom{52}{5} = \frac{52!}{5!(52-5)!} = \frac{52!}{5! \cdot 47!} = 2,598,960 \text{ different poker hands}\]

There are only four possible poker hands that would contain a royal flush, one of each suit. This means that the probability is:
\[P(\text{royal flush hand}) = \frac{4}{\binom{52}{5}} = \frac{4}{2,598,960} \approx 1.539 \cdot 10^{-6} \equiv 0.00154\%\]

\section*{Question 2}
For the Monty Hall puzzle presented in the problem, we consider four doors. You have a $\frac{1}{4}$, or 25\%, chance of selecting the right door without changing doors. 

If we were to change doors, we would have to consider the remaining probability, $\frac{3}{4}$ or 75\%. However, we also have to consider the fact that the host also uncovers one of the other losing doors before you change doors, meaning that the remaining $\frac{3}{4}$ is split between the two remaining doors. In mathematical form, we can represent this by denoting $D_1, D_2, D_3, D_4$ as the doors in the puzzle. We can imagine that we chose $D_1$ as an incorrect door and any one of the others doors is revealed to be an incorrect door as well.
\[P(\text{right door after changing}) = \frac{P(D_2) + P(D_3) + P(D_4)}{2} = \frac{\frac{1}{4} + \frac{1}{4} + \frac{1}{4}}{2} = \frac{\frac{3}{4}}{2} = \frac{3}{8}\]

\section*{Question 3}
There are 6, or $_{3}P_{3}$, total permutations for the set \{1,2,3\}.

\renewcommand{\labelenumi}{(\alph{enumi})}

\begin{enumerate}
    \item There are only two possible events where 1 precedes 3: \{1,2,3\} and \{1,3,2\}. Therefore, the probability that the randomly selected permutation contains a set where 1 precedes 3 is $\frac{1}{3}$, or 33\%.
    \item There are only two possible events where 3 precedes both 1 and 2: \{3,1,2\} and \{3,2,1\}. Therefore, the probability that the randomly selected permutation contains a set where 3 precedes both 1 and 2 is $\frac{1}{3}$, or 33\%.
\end{enumerate}

\section*{Question 4}
There are 26!, or about $4.03 \cdot 10^{26}$, total permutations of the 26 lowercase letters.

\begin{enumerate}
    \item To find the probability that the randomly set of lowercase letters contains the first 13 letters in order, we only need to consider the amount of permutations of the last 13 letters because our condition will be equally likely in each of these permutations. Therefore, the probability is:
    
    \[P(\text{first 13 in order)} = \frac{_{13}P_{13}}{26!} = \frac{13!}{26!} = \frac{1}{13!} = 1.606 \cdot 10^{-10}\]
    
    \item To find the probability that the randomly set of lowercase letters contains a and z next to each other, we can simplify what we are permuting on. In other words, we can think of the set as permuting 25 objects, the letters b through y and ``az". In combination with the 2 permutations of ``az", we then find that there are 25! permutations of the 25 objects set described previously. Thus the probability is:
    
    \[P(\text{a and z are adjacent}) = \frac{2 \cdot 25!}{26!} = \frac{1}{13} \approx 0.077 \equiv 7.7\%\]
    
    \item The probability that the randomly permuted set doesn't contain a and z adjacent to one another is just to opposite of the answer in part (c):
    \[P(\text{a and z are not adjacent}) = 1 - P(\text{a and z are adjacent}) = 1 - \frac{1}{13} = \frac{12}{13} \approx 0.923 \equiv 92.3\%\]
\end{enumerate}

\section*{Question 5}
We know that events $E$ and $F$ are independent if and only if $p(E \cap F) = p(E)p(F)$ from the class notes. We can list all of the possible bit strings of length three.

\setlength{\tabcolsep}{16pt}
\renewcommand{\arraystretch}{1.25}
\begin{table}[H]
    \begin{tabular}{c c c}
         000 & 011 & 110 \\
         001 & 100 & 111 \\
         010 & 101 &  \\
    \end{tabular}
\end{table}
From the list above, we can see that half of the bit strings have an odd number of 1's (\{001, 010, 100, 111\}) and half of them starts with 1 (\{100, 101, 110, 111\}). Therefore, $p(E) = \frac{1}{2}$ and $p(F) = \frac{1}{2}$. In addition, we can observe that two of the bit strings listed previously fits the criteria of when events $E$ and $F$ occur at the same time ($E \cap F$ : \{100, 111\}). Therefore, $p(E \cap F) = \frac{1}{4}$. Finally, we can see if the conditions for independent events is true in this situation. We observe that:

\[p(E)p(F) = \left(\frac{1}{2}\right)\left(\frac{1}{2}\right) = \frac{1}{4} = p(E \cap F)\]
Therefore, $E$ and $F$ are independent.

\end{document}