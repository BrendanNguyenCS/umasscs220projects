\documentclass[letterpaper, 12pt]{article}
\usepackage{nopageno} % For removing page numbers
\usepackage[utf8]{inputenc}
\usepackage{titling} % For positioning of title preamble
\usepackage[margin=0.75in]{geometry} % For margin width setting
\usepackage{comment} % For block commenting
\usepackage{enumitem} % For list styling
\usepackage{float} % For table positioning
\usepackage{parskip} % For automatic paragraph spacing/formatting
% For automatic table numbering
\usepackage{array}
\newcounter{rowno}
\setcounter{rowno}{0}
% Change default font to sans font
\renewcommand{\familydefault}{\sfdefault}

% Move title area to the top of the page
\setlength{\droptitle}{-4em}
\addtolength{\droptitle}{-4pt} 
\setlength{\tabcolsep}{12pt}
\renewcommand{\arraystretch}{1.25}
% Increase width between table columns
\setlength{\tabcolsep}{1cm}
% Redefine list numbering system
\renewcommand{\labelenumi}{(\alph{enumi})}
\renewcommand{\labelenumii}{\roman{enumii}.}


\title{CS220 Discrete Math - Homework \#2}
\author{Brendan Nguyen - \texttt{brendan.nguyen001@umb.edu}}
\date{February 10, 2022}

\begin{document}

\maketitle

\section*{Question 1}

\begin{enumerate}
    \item For the domain of $x$ being all students in CS220
    \begin{enumerate}
        \item $\forall x(P(x))$ represents the statement, "Everyone in CS220 has a cellphone."
        \item $\exists x(Q(x))$ represents the statement, "Somebody in CS220 can solve quadratic equations."
        \item $\exists x(\neg R(x))$ represents the statement, "Somebody in CS 220 does not want to be rich."
    \end{enumerate}
    \item For the domain of $x$ being all students and that $C(x)$ is a predicate for "$x$ is in CS220" 
    \begin{enumerate}
        \item $\forall x(C(x) \to P(x))$ represents the statement, "Everyone in CS220 has a cellphone."
        \item $\exists x(C(x) \land Q(x))$ represents the statement, "Somebody in CS220 can solve quadratic equations."
        \item $\exists x(C(x) \land \neg R(x))$ represents the statement, "Somebody in CS 220 does not want to be rich."
    \end{enumerate}
\end{enumerate}

\section*{Question 2}

The argument with the given premises and the conclusion $r$ is valid as shown below:
\begin{table}[H]
    \begin{tabular}{>{\stepcounter{rowno}\therowno. }l l}
         \multicolumn{1}{l}{\textbf{Step}} & \textbf{Reason} \\
         $(p \land t) \to (r \lor s)$ & Premise\\
         $\neg p \lor \neg t \lor r \lor s$ & Logical equivalence of (1)\\
         $s \lor (\neg p \lor \neg t \lor r)$ & Commutative and Associative laws on (2)\\
         $\neg s$ & Premise\\
         $\neg p \lor \neg t \lor r$ & Disjunctive syllogism on (3) and (4)\\
         $u \to p$ & Premise\\
         $\neg u \lor p$ & Logical equivalence of (6)\\
         $p \lor \neg u$ & Commutative law on (7)\\
         $\neg u \lor \neg t \lor r$ & Resolution rule on (5) and (7)\\
         $\neg (u \land t) \lor r$ & DeMorgan's Law on (9)\\
         $(u \land t) \to r$ & Logical equivalence of (10)\\
         $q \to (u \land t)$ & Premise\\
         $q \to r$ & Hypothetical syllogism on (12) and (11)\\
         $q$ & Premise\\
         $r$ & Modus ponens on (14) and (13)
    \end{tabular}
\end{table}


\section*{Question 3}
Let the conditional statement $S$ be $(p \to q) \to q$, then we can use a truth table to see the truth values.

\begin{table}[H]
    \centering
    \begin{tabular}{|c c|c|c|}
        \hline
         $p$ & $q$ & $p \to q$ & $(p \to q) \to q$  \\\hline
         T & T & T & T\\\hline
         T & F & F & T\\\hline
         F & T & T & T\\\hline
         F & F & T & F\\\hline
    \end{tabular}
    \caption{Truth table for $(p \to q) \to q$}
    \label{tab:1}
\end{table}

In the statement above, $p$ represents the statement, "$S$ is true," and $q$ represents the statement, "Unicorns live." If $S$ is a proposition (in other words $p \to q$ is $\texttt{true}$), then $q$ is $\texttt{true}$, meaning that "Unicorns live." If $S$ is not a proposition, then it has both $\texttt{true}$ and $\texttt{false}$ values. We know that if $S$ is $\texttt{true}$, then $p$ is $\texttt{true}$. Both $\texttt{true}$ and $\texttt{false}$ for $q$ still results in $\texttt{true}$ for $S$. If $S$ is $\texttt{false}$, then $(p \to q)$ is $\texttt{true}$, $p$ and $q$ are both $\texttt{false}$. Due to the fact that we can't determine the truth value of $S$, we can determine that $S$ is not a proposition by the definition that a statement can have a $\texttt{true}$ or $\texttt{false}$ value but not both.

\section*{Question 4}
The statement, "No one has more than two grandmothers," can be rewritten as, "There does not exists three different people that are grandmothers of the same person". We can translate this statement into a predicate as shown below:

\[
\forall x \Bigg[ \neg \exists w \exists y \exists z [ G(w,x) \land G(y,x) \land G(z,x) \land (w \neq y \neq z)] \> \Bigg]
\]

where $x$ is a person and $w$, $y$, and $z$ are grandmothers of person $x$.

\section*{Question 5}
Because $A$ is a nonempty set, you can say that $a \in A$. In order to prove if $B = C$, we denote a variable $x$ such that $x \in B$. Then by set product, we can say that $\langle a,x \rangle \in A \times B$. Using the original problem statement, $A \times B = A \times C$, we can then state that $\langle a,x \rangle \in A \times C$ and $x \in C$ using set product again. Therefore, $B$ is included in $C$. Now, since $A$ is a subset of $B$ because every element of $A$ is also an element of $B$, we can say that $B$ is a subset of $C$ and $C$ is a subset of $B$. Therefore, by this definition, we can state that $B = C$ since $B \subseteq C$ and $C \subseteq B$.

\end{document}