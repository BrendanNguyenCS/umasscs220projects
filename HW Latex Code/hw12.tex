\documentclass[11pt]{article}
\usepackage{nopageno} % For removing page numbers
\usepackage[utf8]{inputenc}
\usepackage{titling} % For positioning of title preamble
\usepackage[margin=1in]{geometry} % For margin width setting
\usepackage{comment} % For block commenting
\usepackage{enumitem} % For list styling
\usepackage{float} % For table positioning
% For math equation formatting
\usepackage{amsmath, amssymb, relsize}
% For automatic paragraph spacing/formatting
\usepackage{parskip}
% For side by side figures
\usepackage{multicol}
\usepackage{makecell}
% For colors
\usepackage[dvipsnames]{xcolor}


% Move title area to the top of the page
\setlength{\droptitle}{-4em}
\addtolength{\droptitle}{-4pt} 
% \setlength{\tabcolsep}{12pt}
\renewcommand{\arraystretch}{1.25}
% Disable paragraph indenting
\setlength{\parindent}{0pt}
% Change default font to sans font
\renewcommand{\familydefault}{\sfdefault}
% Change default vertical spacing for align environments
\setlength{\jot}{7pt}

\title{CS220 Discrete Math - Homework \#12}
\author{Brendan Nguyen - \texttt{brendan.nguyen001@umb.edu}}
\date{May 5, 2022}

\begin{document}

\maketitle

\section*{Question 1}
The matrix representing $R^2$ can be found by multiplying the matrix that represents $R$ by itself.

\[
M_{R^2} = \left[
\begin{array}{c c c}
    0 & 1 & 1 \\
    1 & 1 & 0 \\
    1 & 0 & 1
\end{array}
\right]
\cdot 
\left[
\begin{array}{c c c}
    0 & 1 & 1 \\
    1 & 1 & 0 \\
    1 & 0 & 1
\end{array}
\right]
= 
\left[
\begin{array}{c c c}
    1 & 1 & 1 \\
    1 & 1 & 1 \\
    1 & 1 & 1
\end{array}
\right]
\]

\section*{Question 2}


\section*{Question 3}
To show that $R$ is an equivalence relation, we must show that $R$ is reflexive, symmetric, and transitive. 

We can say that $R$ is \textbf{reflexive} because $((a,b), (a,b)) \in R$ since $ab = ab$. 

If $ab = bc$, then $cb = da$ and $((c,d), (a,b)) \in R$, meaning that $R$ is \textbf{symmetric}. We then denote $(e,f)$ as another ordered pair in the relation. 

If they were to follow the previously stated rule, then $((a,b), (c,d)) \in R$ and $((c,d), (e,f)) \in R$ which means that $ad = bc$ and $cf = de$. Multiplying this system of equations together results in $acdf = bcde$. Simplifying this equation results in $af = be$. Therefore $((a,b), (e,f)) \in R$ meaning that $R$ is also \textbf{transitive}.

With all of this in mind, we can say that $R$ is an equivalence relation because it is reflexive, symmetric, and transitive.

\section*{Question 4}


\section*{Question 5}


\end{document}