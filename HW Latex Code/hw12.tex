\documentclass[letterpaper, 12pt]{article}
\usepackage{nopageno} % For removing page numbers
\usepackage[utf8]{inputenc}
\usepackage{titling} % For positioning of title preamble
\usepackage[margin=0.75in]{geometry} % For margin width setting
\usepackage{comment} % For block commenting
\usepackage{enumitem} % For list styling
\usepackage{float} % For table positioning
% For math equation formatting
\usepackage{amsmath, amssymb, relsize}
% For automatic paragraph spacing/formatting
\usepackage{parskip}
% For side by side figures
\usepackage{multicol}
\usepackage{makecell}
% For colors
\usepackage[dvipsnames]{xcolor}


% Move title area to the top of the page
\setlength{\droptitle}{-4em}
\addtolength{\droptitle}{-4pt} 
% \setlength{\tabcolsep}{12pt}
\renewcommand{\arraystretch}{1.25}
% Disable paragraph indenting
\setlength{\parindent}{0pt}
% Change default font to sans font
\renewcommand{\familydefault}{\sfdefault}
% Change default vertical spacing for align environments
\setlength{\jot}{7pt}
% Redefine list numbering system
\renewcommand{\labelenumi}{(\alph{enumi})}

\title{CS220 Discrete Math - Homework \#12}
\author{Brendan Nguyen - \texttt{brendan.nguyen001@umb.edu}}
\date{May 5, 2022}

\begin{document}

\maketitle

\section*{Question 1}
The matrix representing $R^2$ can be found by multiplying the matrix that represents $R$ by itself.

\[
M_{R^2} = M_R \circ M_R = 
\left[
\begin{array}{c c c}
    0 & 1 & 1 \\
    1 & 1 & 0 \\
    1 & 0 & 1
\end{array}
\right]
\left[
\begin{array}{c c c}
    0 & 1 & 1 \\
    1 & 1 & 0 \\
    1 & 0 & 1
\end{array}
\right]
= 
\left[
\begin{array}{c c c}
    1 & 1 & 1 \\
    1 & 1 & 1 \\
    1 & 1 & 1
\end{array}
\right]
\]

\section*{Question 2}

For these problems, we will be finding $R$, which is the smallest relation that contains the given relation $\{(1, 2),(1, 4),(3, 3),(4, 1)\}$.
\begin{enumerate}
    \item In order for $R$ to be \textbf{reflexive}, we know that $(1,1)$, $(2,2)$, $(3,3)$, and $(4,4)$ must be in the relation.
    
    In order for $R$ to be \textbf{transitive}, we can see a number of things about the given relation. First, since $(1,4)$ and $(4,1)$ are in the set, 1 relates to itself (in other words, $(1,1)$ must be in the relation. Similarly, $(4,4)$ must be in the set because $(4,1)$ and $(1,4)$ are in the given relation. Finally, $(4,2)$ must also be in the relation because $(4,1)$ and $(1,2)$ are in the given relation.
    
    Therefore, the smallest relation that is reflexive and transitive is:
    \[R: \{(1,1), (2,2), (3,3), (4,4), (1,2), (4,1), (1,4), (4,2)\}.\]
    
    \item In order for $R$ to be \textbf{symmetric}, we know that $(2,1)$ must be in the relation because $(1,2)$ is in the relation as well. In addition, $(4,1)$ must be in the relation because $(1,4)$ is in the relation.
    
    Our analysis of the \textbf{transitive} properties of the relation will be similar to the previous question. $(1,1)$ must be in the relation because $(1,2)$ and $(2,1)$ are in the relation. Similarly, $(4,4)$ exists because $(4,1)$ and $(1,4)$ are in the relation. Because $(2,1)$ and $(1,4)$ are in the relation, $(2,4)$ must also exist in this relation. Similarly, $(4,2)$ exists since $(4,1)$ and $(1,2)$ exists in the relation. Finally, $(2,2)$ must exist because $(2,4)$ and $(4,2)$ exist.
    
    Therefore, the smallest relation that is symmetric and transitive is:
    \[R: \{(1,1), (2,2), (3,3), (4,4), (1,2), (1,4), (2,1), (2,4), (4,1), (4,2)\}\]
    
    \item We can use the same \textbf{reflexive} reasoning as in part (a) to say that $(1,1)$, $(2,2)$, $(3,3)$, and $(4,4)$ are in $R$.
    
    For $R$ to be \textbf{symmetric}, $(2,1)$ must be in $R$ because $(1,2)$ is in the relation. Also, $(4,1)$ must be in $R$ because $(1,4)$ is in the relation.
    
    For $R$ to be \textbf{transitive}, $(4,2)$ and $(2,4)$ are in the relation. This is due to $(4,1)$ and $(1,2)$, and $(2,1)$ and $(1,4)$ are in the relation, respectively.
    
    Therefore, the smallest relation that is reflexive, symmetric, and transitive is:
    \[R: \{(1,1), (2,2), (3,3), (4,4), (1,2), (1,4), (2,1), (2,4), (4,1), (4,2)\}\]
\end{enumerate}

\section*{Question 3}
To show that $R$ is an equivalence relation, we must show that $R$ is reflexive, symmetric, and transitive.

We can say that $R$ is \textbf{reflexive} because $((a,b), (a,b)) \in R$ since $ab = ba$. 

If $ad = bc$, then $cb = da$ and $((c,d), (a,b)) \in R$, meaning that $R$ is \textbf{symmetric}.

We then denote $(e,f)$ as another ordered pair in the relation. If they were to follow the previously stated rule, then $((a,b), (c,d)) \in R$ and $((c,d), (e,f)) \in R$ which means that $ad = bc$ and $cf = de$. Multiplying this system of equations together results in $acdf = bcde$. Simplifying this equation results in $af = be$. Therefore $((a,b), (e,f)) \in R$ meaning that $R$ is also \textbf{transitive}.

With all of this in mind, we can say that $R$ is an equivalence relation because it is reflexive, symmetric, and transitive.

\section*{Question 4}
For each of the matrices, they are transitive if all of the diagonals are all 1s. To check for symmetry, we will turn the rows of the matrix into columns and check if the matrix is still the same. To check if it is transitive, the matrix representing $R^2$ should be the same as the matrix that represents $R$ (in other words, $M_R \circ M_R = M_R$).

\begin{enumerate}
    \item This relation is \textbf{reflexive} because all of its diagonals are 1's. The matrix to check for symmetry is:
    
    \[
    \left[
    \begin{array}{c c c}
         1 & 0 & 1 \\
         1 & 1 & 1 \\
         1 & 1 & 1
    \end{array}
    \right]
    \]
    
    Because this matrix is not the same as the original matrix, we can say that $R$ is \textbf{not symmetric}.
    
    Because the matrix is not symmetric, we can say that the matrix is not a equivalence relation without checking for transitivity.
    
    \item This relation is \textbf{reflexive} because all of its diagonals are 1's. The matrix to check for symmetry is:
    
    \[
    \left[
    \begin{array}{c c c c}
         1 & 0 & 1 & 0 \\
         0 & 1 & 0 & 1 \\
         1 & 0 & 1 & 0 \\
         0 & 1 & 0 & 1
    \end{array}
    \right]
    \]
    
    This matrix is the same as the original matrix so we can say that it is \textbf{symmetric}. To check for transitivity, we have the following:
    
    \[
    M_{R^2} = M_R \circ M_R =
    \left[
    \begin{array}{c c c c}
         1 & 0 & 1 & 0 \\
         0 & 1 & 0 & 1 \\
         1 & 0 & 1 & 0 \\
         0 & 1 & 0 & 1
    \end{array}
    \right]
    \left[
    \begin{array}{c c c c}
         1 & 0 & 1 & 0 \\
         0 & 1 & 0 & 1 \\
         1 & 0 & 1 & 0 \\
         0 & 1 & 0 & 1
    \end{array}
    \right]
    =
    \left[
    \begin{array}{c c c c}
         1 & 0 & 1 & 0 \\
         0 & 1 & 0 & 1 \\
         1 & 0 & 1 & 0 \\
         0 & 1 & 0 & 1
    \end{array}
    \right]
    \]
    
    Because $M_{R^2}$ is the same matrix as $M_R$, we can say that $M_R$ is a \textbf{transitive} matrix.
    
    Therefore, the matrix that represents $R$ represents an equivalence relation because it's reflexive, symmetric, and transitive.
    
    \item This relation is \textbf{reflexive} because all of its diagonals are 1's. The matrix to check for symmetry is:
    
    \[
    \left[
    \begin{array}{c c c c}
         1 & 1 & 1 & 0 \\
         1 & 1 & 1 & 0 \\
         1 & 1 & 1 & 0 \\
         0 & 0 & 0 & 1
    \end{array}
    \right]
    \]
    
    This matrix is the same as the original matrix so we can say that it is \textbf{symmetric}. To check for transitivity, we have the following:
    
    \[
    M_{R^2} = M_R \circ M_R = 
    \left[
    \begin{array}{c c c c}
         1 & 1 & 1 & 0 \\
         1 & 1 & 1 & 0 \\
         1 & 1 & 1 & 0 \\
         0 & 0 & 0 & 1
    \end{array}
    \right]
    \left[
    \begin{array}{c c c c}
         1 & 1 & 1 & 0 \\
         1 & 1 & 1 & 0 \\
         1 & 1 & 1 & 0 \\
         0 & 0 & 0 & 1
    \end{array}
    \right]
    =
    \left[
    \begin{array}{c c c c}
         1 & 1 & 1 & 0 \\
         1 & 1 & 1 & 0 \\
         1 & 1 & 1 & 0 \\
         0 & 0 & 0 & 1
    \end{array}
    \right]
    \]
    
    Because $M_{R^2}$ is the same matrix as $M_R$, we can say that $M_R$ is a \textbf{transitive} matrix.
    
    Therefore, the matrix that represents $R$ represents an equivalence relation because it's reflexive, symmetric, and transitive.
\end{enumerate}

\section*{Question 5}
We know that a relation is a partial ordering if it is reflexive, antisymmetric, and transitive. We will test this for each of the relations $M_R$ below.

\begin{enumerate}
    \item $R$ is \textbf{reflexive} because the matrix only contains 1's in the diagonal. 
    
    $R$ is also \textbf{antisymmetric} because $m_{ij} = m_{ij} = 1$ is only true when $i = j$.
    
    We can use $M_{R^2} = M_R \circ M_R = M_R$ to check for transitivity.
    
    \[
    M_{R^2} = M_R \circ M_R =
    \left[
    \begin{array}{c c c}
         1 & 0 & 1 \\
         1 & 1 & 0 \\
         0 & 0 & 1
    \end{array}
    \right]
    \left[
    \begin{array}{c c c}
         1 & 0 & 1 \\
         1 & 1 & 0 \\
         0 & 0 & 1
    \end{array}
    \right]
    =
    \left[
    \begin{array}{c c c}
         1 & 0 & 1 \\
         1 & 1 & 1 \\
         0 & 0 & 1
    \end{array}
    \right]
    \]
    
    Because $M_{R^2} \neq M_R$, $R$ is \textbf{not transitive}.
    
    Therefore, $R$ is not a partial ordering because it is not transitive.
    
    \item $R$ is \textbf{reflexive} because the matrix only contains 1's in the diagonal. 
    
    $R$ is also \textbf{antisymmetric} because $m_{ij} = m_{ij} = 1$ is only true when $i = j$.
    
    We can use $M_{R^2} = M_R \circ M_R = M_R$ to check for transitivity.
    
    \[
    M_{R^2} = M_R \circ M_R =
    \left[
    \begin{array}{c c c}
         1 & 0 & 0 \\
         0 & 1 & 0 \\
         1 & 0 & 1
    \end{array}
    \right]
    \left[
    \begin{array}{c c c}
         1 & 0 & 0 \\
         0 & 1 & 0 \\
         1 & 0 & 1
    \end{array}
    \right]
    =
    \left[
    \begin{array}{c c c}
         1 & 0 & 0 \\
         0 & 1 & 0 \\
         1 & 0 & 1
    \end{array}
    \right]
    \]
    
    We can see that $M_{R^2} = M_R$ meaning that $R$ is \textbf{transitive}.
    
    Therefore, $R$ is a partial ordering because it is reflexive, antisymmetric, and transitive.
    
    \item $R$ is \textbf{reflexive} because the matrix only contains 1's in the diagonal.
    
    $R$ is also \textbf{antisymmetric} because $m_{ij} = m_{ij} = 1$ is only true when $i = j$.
    
    We can use $M_{R^2} = M_R \circ M_R = M_R$ to check for transitivity.
    
    \[
    M_{R^2} = M_R \circ M_R =
    \left[
    \begin{array}{c c c c}
         1 & 0 & 1 & 0 \\
         0 & 1 & 1 & 0 \\
         0 & 0 & 1 & 1 \\
         1 & 1 & 0 & 1
    \end{array}
    \right]
    \left[
    \begin{array}{c c c c}
         1 & 0 & 1 & 0 \\
         0 & 1 & 1 & 0 \\
         0 & 0 & 1 & 1 \\
         1 & 1 & 0 & 1
    \end{array}
    \right]
    =
    \left[
    \begin{array}{c c c c}
         1 & 0 & 1 & 1 \\
         0 & 1 & 1 & 1 \\
         1 & 1 & 1 & 1 \\
         1 & 1 & 1 & 1
    \end{array}
    \right]
    \]
    
    Because $M_{R^2} \neq M_R$, $R$ is \textbf{not transitive}.
    
    Therefore, $R$ is not a partial ordering because it is not transitive.
\end{enumerate}


\end{document}