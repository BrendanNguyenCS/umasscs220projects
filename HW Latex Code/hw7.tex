\documentclass[letterpaper, 12pt]{article}
\usepackage{nopageno} % For removing page numbers
\usepackage[utf8]{inputenc}
\usepackage{titling} % For positioning of title preamble
\usepackage[margin=0.75in]{geometry} % For margin width setting
\usepackage{comment} % For block commenting
\usepackage{enumitem} % For list styling
\usepackage{float} % For table positioning
% For math equation formatting
\usepackage{amsmath}
\usepackage{relsize}
\usepackage{amssymb}
\usepackage(amsfonts)
% For automatic paragraph spacing/formatting
\usepackage{parskip}
% For side by side figures
\usepackage{multicol}
\usepackage{makecell}
% For colors
\usepackage[dvipsnames]{xcolor}
% For code
\usepackage{listings}
\lstdefinestyle{mystyle}{
    basicstyle=\ttfamily\small,
    breakatwhitespace=false,         
    breaklines=true,                 
    captionpos=b,                    
    keepspaces=true,                 
    numbersep=5pt,                  
    showspaces=false,                
    showstringspaces=false,
    showtabs=false,                  
    tabsize=2
}
\lstset{style=mystyle}

% Move title area to the top of the page
\setlength{\droptitle}{-4em}
\addtolength{\droptitle}{-4pt} 
% \setlength{\tabcolsep}{12pt}
\renewcommand{\arraystretch}{1.25}
% Disable paragraph indenting
\setlength{\parindent}{0pt}
% Times New Roman font
\usepackage{times}
\renewcommand{\ttdefault}{lmtt}
% Change default vertical spacing for align environments
\setlength{\jot}{7pt}

\title{CS220 Discrete Math - Homework \#7}
\author{Brendan Nguyen - \texttt{brendan.nguyen001@umb.edu}}
\date{March 24, 2022}

\begin{document}

\maketitle

\section*{Question 1}
We will use induction to prove that 6 divides $n^3 - n$ for every nonnegative integer $n$. We can first define a temporary function $F(n)$ that is defined by the statement ``6 divides $n^3 - n$".

\textbf{Basis step: } $n = 0$\\
$0^3 - 0 = 0$ $\&$ 6 divides 0 $\therefore$ $F(n)$ is true when $n = 0$

\textbf{Inductive step: } $F(n) \to F(n + 1)$\\
We can assume that $F(n)$ is true for some positive integer $n$. We can calculate $F(n + 1)$ as shown below:
\begin{align*}
    (n + 1)^3 - (n + 1) &= n^3 + 3n^2 + 3n + 1 - n - 1\\
    &= (n^3 - n) + (3n^2 + 3n)\\
    &= (n^3 - n) + 3n(n+1)
\end{align*}

Next, we can use features of numbers to help with our proof. We know that exactly one of any two consecutive integers will be even. Therefore, we can say that 2 divides $n(n + 1)$ and $3n(n + 1)$. We can also say that both 3 and 6 divide $3n(n+1)$. 

By inductive hypothesis, we know that 6 divides $n^3 - n$. The previous two statements allow us to state that 6 divides $(n^3 - n) + 3n(n+1)$. Since $(n^3 - n) + 3n(n+1) = (n + 1)^3 - (n + 1)$, we can state that 6 also divides $(n + 1)^3 - (n + 1)$, in order words 6 divides $F(n + 1)$.

With everything in mind, we can say that $F(n) \to F(n + 1)$.

\section*{Question 2}
We can mathematical induction to prove that $AB^n = B^nA$ for all nonnegative integer $n$ given that $AB = BA$. 

Firstly, we can denote a function $F(n)$ that is defined by the given statement above. Then we can prove that the function is true for $n = 1$.
\begin{align*}
    AB^1 &= AB\\
    B^1A &= BA
\end{align*}

The statements above signify each side of the inequality. Because both sides are equal to each other, we can say that $F(n)$ is true for $n = 1$. Next, we assume that $F(x)$ is true and then prove that $F(n + 1)$ is true. In other words, if $AB = BA$, then $AB^x = B^xA$. By proving if $F(x + 1)$ is true, we will find out if $AB^{x + 1} = B^{x + 1}A$ given that $AB = BA$.
\allowdisplaybreaks
\begin{align*}
    AB^{x + 1} &= A \cdot (B^xB)\\
    &= (AB^x) \cdot B\\
    &= (B^x \cdot A)B\\
    &= B^x(AB)\\
    &= B^x(BA)\\
    &= (B^xB)A\\
    &= B^{x + 1}A
\end{align*}

Similar to before, we can say that $F(x + 1)$ is true when $F(x)$ is true. Finally, by mathematical induction, we can say that $F(n)$ is true for all nonnegative integers $n$.

\section*{Question 3}
The main issue with this induction proof is the basis step. By claiming that, "$a^0 = 1$ is true by the definition of $a^0$," they are making an incorrect assumption. In order for the inductive step to work correctly, the denominator used must be one.
\[a^{k + 1} = \frac{a^k \cdot a^k}{a^{k - 1}}\]

The denominator hints that you must have $k \geq 1$ but we need to start the induction at $k = 0$. However, when we start at $k = 0$, then we get
\[a^1 = \frac{a^0 \cdot a^0}{a^{-1}}\]

But, we don't know for certain that $a^{-1} = 1$.

\section*{Question 4}
The set of all bit strings that are palindromes can be defined as:

\begin{itemize}
    \item The empty string and the bit strings 0 and 1 are in the set
    \item If a string $s$ is in the set, then 0$s$0 and 1$s$1
    \item A string $s$ is in the set if it can be constructed using the two previously stated rules
\end{itemize}

\section*{Question 5}
We can use the given equality $a^{2^{n+1}} = (a^{2^{n}})^2$ to write the recursive algorithm below:

\begin{lstlisting}[mathescape=true]
function pow(x : a real number, n : a positive integer) {
    if (n = 1) {
        return $x^2$
    }
    else {
        return pow(x, n - 1)$^2$
    }
}
\end{lstlisting}


\end{document}