\documentclass[11pt]{article}
\usepackage[utf8]{inputenc}
\usepackage{titling} % For positioning of title preamble
\usepackage[margin=1in]{geometry} % For margin width setting
\usepackage{comment} % For block commenting

% Move title area to the top of the page
\setlength{\droptitle}{-4em}
\addtolength{\droptitle}{-4pt} 
\setlength{\tabcolsep}{12pt}
\renewcommand{\arraystretch}{1.25}
% Disable paragraph indenting
\setlength{\parindent}{0pt}


\title{CS220 Discrete Math - Homework \#2}
\author{Brendan Nguyen - \texttt{brendan.nguyen001@umb.edu}}
\date{February 10, 2022}

\begin{document}

\maketitle

\section*{Question 1}
\renewcommand{\labelenumi}{(\alph{enumi})}
\renewcommand{\labelenumii}{\roman{enumii}.}

\begin{enumerate}
    \item For the domain of $x$ being all students in CS220
    \begin{enumerate}
        \item $\forall x(P(x))$ represents the statement, "Everyone in CS220 has a cellphone."
        \item $\exists x(Q(x))$ represents the statement, "Somebody in CS220 can solve quadratic equations."
        \item $\exists x(\neg R(x))$ represents the statement, "Somebody in CS 220 does not want to be rich."
    \end{enumerate}
    \item For the domain of $x$ being all students and that $C(x)$ is a predicate for "$x$ is in CS220" 
    \begin{enumerate}
        \item $\forall x(C(x) \to P(x))$ represents the statement, "Everyone in CS220 has a cellphone."
        \item $\exists x(C(x) \wedge Q(x))$ represents the statement, "Somebody in CS220 can solve quadratic equations."
        \item $\exists x(C(x) \wedge \neg R(x))$ represents the statement, "Somebody in CS 220 does not want to be rich."
    \end{enumerate}
\end{enumerate}

\section*{Question 2}


\section*{Question 3}


\section*{Question 4}


\section*{Question 5}

\end{document}